\documentclass[a4paper, 11pt]{article}

\usepackage[T1]{fontenc}
\usepackage[utf8]{inputenc}
\usepackage{xcolor}
\usepackage{listings}
\usepackage{url}

\lstset{basicstyle=\footnotesize\ttfamily}

\begin{document}
\title{Benchmark sur OcsigenServer}
\author{Romain Calascibetta}

\maketitle

\section{Introduction}

Le benchmarking de  serveur  HTTP  est  un  critère  de sélection important pour
l'utilisateur final.  Ainsi,  le résultat que  peut proposer un benchmarking est
très  important.   Cependant,  le  benchmarking,  et  tout  particulièrement  le
benchmarking  de  serveur  HTTP  n'est  pas  si  simple.   En  effet,   utiliser
basiquement des outils  tel que Apache Benchmark ou  Httperf peuvent vous amener
à tester  les performances de  votre serveur physique  ou de votre  réseau et le
résultat n'en est pas plus intéressant.

Dans ce document,  nous allons voir  le contexte de benchmarking d'OcsigenServer
et comparer sa version 2.4.0 avec  la futur version 3.0.0 utilisant la librairie
Cohttp.  Nous continuerons  ce benchmarking avec  Nginx et Apache  pour avoir un
ordre  d'idée   sur  les   performances  d'OcsigenServer   par  rapport   à  ces
concurrents.

\section{Contexte Physique}

Le  contexte  physique  d'un benchmarking  sur  un  serveur  HTTP  est  un point
important car il doit être en  phase avec une situation de déploiement générale.
Lancer les outils de  test  sur  le  même  ordinateur  que  le serveur n'est pas
efficient    surtout    si    ce    dernier    utilise    des    techniques   de
\emph{multi-threading}.  Dans cette situation notre test consistera à évaluer le
\emph{scheduling} de notre système entre l'outil de test et le serveur.

\subsection{Test sur la concurrence}

Ce facteur qui  est déterminant dans une situation  de déploiement générale.  En
effet,  à moins  que votre site  internet ne soit  visiter que par  une seule et
unique  personne,  la question  de la  concurrence  des  requêtes  est  un point
central dans le benchmarking de serveur HTTP.

Dans une situation lambda,  la concurrence ne s'applique pas qu'à une seule page
internet mais sur l'ensemble du site (image, script JavaScript,  CSS, etc.) mais
ce point évaluera  plutôt les performances du  site web que le  serveur HTTP lui
même.  Cela ne  nous concerne pas (ou  pourrait nous concerner  si nous voulions
faire un benchmarking sur Eliom).

Dans nos  objectifs,  nous voulons tester  la vitesse  pure d'OcsigenServer.  Et
l'une des questions à laquelle  nous devons répondre est:  \guillemotleft Est ce
que le  serveur est capable de  gérer plusieurs  connexions simultanément  et si
oui, quelle est sa limite ?\guillemotright

D'un  point de  vu technique,  un  tel test  peut  être  limité  par  le serveur
physique lui-même.  En  effet,  une distribution standard tel  que Debian impose
des limites arbitraires sur le serveur et ses processus.  Nous devons autant sur
le client que  sur le serveur enlever  ces limites pour que  le benchmarking n'y
soit que très peu influencé.

\subsection{Test sur le réseau}

Le  réseau peut  être un  frein au  benchmarking.  En effet,  les  tests peuvent
saturer le réseau et le client peut être sujet à un goulot d'étranglement.  Pour
que le test soit efficient,  seul le  serveur peut être un goulot d'étranglement
(et dans cette situation, nous arrivons à des limitations physiques).  Ainsi, il
est préférable d'avoir  au  moins  2  clients  physiques  afin  d'éviter une tel
situation.  Un goulot d'étranglement peut s'évaluer  à l'aide de l'I/O du réseau
sur le client. L'outil Httperf vous notifiera de l'approche d'une telle limite.

La  charge  CPU  n'est  pas  un  facteur  déterminant  pour  repérer  un  goulot
d'étranglement.  En effet Httperf  peut avoir une forte utilisation  du CPU même
quand il dispose de capacités inutilisées.

Un test  sur un réseau local  est préférable au  vu de  la complexité  du réseau
internet.  Nous pourrions supposer que celui-ci ne rajoute que du \textsl{bruit}
au benchmark et puisse se  simplifier en un facteur constant cependant,  n'ayant
pas souvent pas connaissance de  l'architecture réseau de votre fournisseur,  il
serait dangereux de considérer ce point comme tel.

L'objectif est  donc d'abstraire au mieux  la contrainte réseau  pour tester les
performances pures  du serveur  HTTP.  Celle-ci existera  toujours mais  si elle
intervient sur  le serveur  \textbf{uniquement},  cela valide  tout les  tests à
charge  inférieur comme  valide (n'ayant  pas été  influencé par  une quelconque
limite physique du  côté  du  client  et  du  serveur).  Il  est donc préférable
d'avoir un  serveur connecté à 2  clients physiques à l'aide  d'un switch réseau
(ce dernier peut être connecté à un  ou plusieurs cartes réseaux du serveur pour
augmenter la bande passante dont le serveur dispose).

\subsection{Test sur la taille des données}

Ce  dernier  test  n'est  pas  forcément  le  test  le  plus  efficient  pour un
benchmarking d'un  serveur HTTP.  En effet,  il peut  amener à ne  tester que la
vitesse de lecture de votre  système.  Cependant,  il faut prouver que appliquer
ce test n'implique que les limitations  physiques de ce système.  Pour ce faire,
il faut tester  le serveur HTTP avec des  serveurs de plus en plus  gros (et que
les autres paramètres  soient  invariants)  et  si  on  dénote une différence de
performance pouvant se  factoriser à un facteur constant,  c'est  que le serveur
HTTP n'est plus  qu'assujettis  aux  limitations  systèmes  sur  la lecture d'un
fichier\footnotemark[1].

\footnotetext[1]{On  devrait  prendre  en  considération aussi le
réseau  dans   ce   test.   Cependant,   si   vous   respecter  les  contraintes
expliquées   ci-dessus   et  que   vous   faites   bien   attention   au  goulot
d'étranglement, ce test devrait rester vrai.}

\subsection{Configuration}

Nous avons  expliqué juste  avant que le  système d'exploitation  pouvait limité
les  tests.   En  effet,   celui-ci  est  généralement  configuré  pour  ce  que
l'utilisateur  final e  besoin ou  pour sauvegarder  les ressources.  Cependant,
nous pouvons configurer le serveur pour dépasser ces limites arbitraires.

Nous allons ici configurer une Debian 7.6 Wheezy.

La  première  configuration est  la  limitation  de  \emph{file  descriptor} par
processus.  Celle-ci est en général limité à 1024 ce qui fait qu'un serveur HTTP
\emph{single-threaded} ne  peut gérer un peu  moins de  1024 connexions  en même
temps. Il s'agit de dépasser cette limite de cette manière:

\begin{lstlisting}
# echo '* soft nofile 20000' >> /etc/security/limits.conf
# echo '* hard nofile 20000' >> /etc/security/limits.conf
\end{lstlisting}

Un autre  problème de limitation  système est le  nombre de port  TCP disponible
qui produit avec les outils tel que Apache Benchmark l'erreur suivante:

\begin{lstlisting}
error: connect() failed: Cannot assign requested address (99)
\end{lstlisting}

Pour  ce problème  nous allons  nous  baser  sur  la  configuration  proposé par
\emph{Performance Scalability  of Multi-Core Web  Server - Bryan  Veal and Annie
Fong} pour le fichier \url{/etc/sysctl.conf}:

\begin{lstlisting}
fs.file-max = 5000000
net.core.netdev_max_backlog = 400000
net.core.optmem_max = 10000000
net.core.rmem_default = 10000000
net.core.rmem_max = 10000000
net.core.somaxconn = 100000
net.core.wmem_default = 10000000
net.core.wmem_max = 10000000
net.ipv4.conf.all.rp_filter = 1
net.ipv4.conf.default.rp_filter = 1
net.ipv4.ip_local_port_range = 1024 65535
net.ipv4.tcp_congestion_control = bic
net.ipv4.tcp_ecn = 0
net.ipv4.tcp_max_syn_backlog = 12000
net.ipv4.tcp_max_tw_buckets = 2000000
net.ipv4.tcp_mem = 30000000 30000000 30000000
net.ipv4.tcp_rmem = 30000000 30000000 30000000
net.ipv4.tcp_sack = 1
net.ipv4.tcp_syncookies = 0
net.ipv4.tcp_timestamps = 1
net.ipv4.tcp_wmem = 30000000 30000000 30000000
\end{lstlisting}

Nous avons  ensuite cette option pour  éviter le  TIME-WAIT pour  les connexions
n'ayant pas le Keep-Alive dans leurs \emph{headers}\footnotemark[1].

\begin{lstlisting}
net.ipv4.tcp_tw_reuse = 1
\end{lstlisting}

Pour  réduire encore  plus le  TIME-WAIT sur  les connexions  TCP,  nous pouvons
utiliser  une  nouvelle  option  qui  n'est  conseillé  de  mettre  que  pour le
benchmarking.  Mettre cette  option sur  la configuration  d'un serveur  mise en
production n'est pas recommandé\footnotemark[2].

\begin{lstlisting}
net.ipv4.tcp_tw_recycle = 1
\end{lstlisting}

Ensuite, il s'agit de charger cette nouvelle configuration à l'aide de:

\begin{lstlisting}
# sysctl -p /etc/sysctl.conf
\end{lstlisting}

Enfin,  au delà de la configuration  système,  nous avons aussi la configuration
du serveur  HTTP qui peut impacter  les performances.  En ce  sens,  même si les
serveurs HTTP comparés  n'ont  pas  les  mêmes  interfaces de configuration,  il
faudrait qu'ils aient plus  ou moins les mêmes paramètres pour  la mise en cache
par exemple.  L'écriture de  log  aussi  à  un  impact  sur  les performances et
celui-ci n'est pas forcément le même entre Nginx et Apache par exemple.

\footnotetext[1]{Le  comportement d'une  requête HTTP  ayant le  Keep-Alive dans
son \emph{header} demande de faire des  tests plus spécifiques (puisque ce point
est en lien avec le technique de \emph{pipeline} propre au protocole HTTP 1.1)}

\footnotetext[2]{En  effet,  cette  option  rend  particulièrement  difficile la
gestion des clients qui sont derrières une même IP (typiquement avec NAT).}

\section{OcsigenServer et Cohttp}

\subparagraph{}
Après  que nous  ayons les  conditions nécessaires  pour un  benchmark efficace,
nous  devons tout  de même  parler des  différences  qu'il  peut  y  avoir entre
OcsigenServer  2.4.0  et   OcsigenServer  3.0.0.   En  effet,   les  différences
techniques de ces  derniers vont certainement influencer les  résultats de leurs
benchmarking respectifs.  Il y a véritablement  deux grosses différences à noter
entre ces 2 versions.

\subparagraph{Pipeline}
La  première  concerne la  gestion  du  \emph{pipeline}.  Celle-ci  concerne une
gestion  de  plusieurs  requêtes  dans  une  même  connexion.  Fasse  à  une tel
situation,  la version 2.4.0 va paralléliser le traitement de ces requêtes alors
que la  version 3.0.0 va traiter  les requêtes  de manière  séquentielle.  De ce
point,  un benchmarking d'un site web dans  sa globalité avec les outils tel que
Siege,  Jmeter ou OpenSTA entre les deux versions devrait dénoter une différence
de performance.

Cependant,  la technique de  \emph{pipeline} est en général peu  utilisé par les
clients  (de  part  sa  complexité de  gestion  pour  les  Firewalls  qui  ne la
supportent  tout simplement  pas).  De  plus,  le  test  de \emph{pipeline},  ne
concerne pas la vitesse pure d'OcsigenServer  sur la charge qu'il peut supporter
sur le nombre de requête simultanée.

\subparagraph{Réutilisation des connexions}
Une  autre  différence  notable  aussi  concerne  l'utilisation  d'OcsigenServer
entant  que  client.   En  effet,   la  version  2.4.0  fait  une  réutilisation
intelligente des  connexions vers d'autres  serveurs alors que  la version 3.0.0
va réinitialiser une  nouvelle connexion pour chaque requête  à destination d'un
autre serveur.

\subparagraph{}
Bien entendu,  ce  dernier point n'entre  pas en compte  dans notre benchmarking
puisque nous  testons avant tout la  vitesse de  \textbf{réponse} du  serveur au
client.  De ce faite, il existe bien des différences techniques entre la version
2.4.0 et la  version 3.0.0 qui ne  sont pas pris en compte  dans ce benchmarking
mais qui  peuvent influencer la vitesse  du serveur  dans des  cas d'utilisation
spécifique.

\section{Les outils de test}

Il existe beaucoup  d'outils  pouvant  tester  un  serveur  HTTP.  Nous en avons
dénoter  3  qui  sont  assez simple  de  déploiement  et  dont  le  résultat est
facilement manipulable.  Chacun présente  des qualités et des  défauts mais cela
ne devrait  pas pour autant les  exclure d'une quelconque  \emph{test suite} car
même si leurs  objectifs  est  le  même,  leurs  comportements  ne  l'est pas et
utiliser Apache Benchmark  par exemple au lieu de  Httperf peut révéler d'autres
cause à la vitesse de traitement des requêtes du serveur.

Il est donc  plus  juste  d'utiliser  tout  ces  outils,  d'essayer  au mieux de
converger leurs résultats pour garantir un plus large contexte d'utilisation.

\subsection{Apache Benchmark}

Apache  Benchmark  est  un  outil  de  test  \emph{single-threaded}  qui  a  été
développé pour tester le serveur HTTP Apache. Il est disponible à l'aide de:

\begin{lstlisting}
# aptitude install apache-utils
\end{lstlisting}

Ensuite,  nous avons  différentes options pour  faire le test  avec cette outil.
Nous allons nous intéresser plus  spécifiquement à l'option de concurrence [-c],
du nombre de requête [-n] et du nombre de seconde du test [-t].

Nous ferons des tests spécifiques avec  celui-ci pour savoir en combien de temps
le  serveur  peut  répondre  pour  [-n]  connexions  simultanées  envoyant  [-c]
requêtes.  Nous ferons le  test  sur  le  fichier  \url{index.html} contenant un
\guillemotleft Hello  World!\guillemotright et sur  un fichier d'1  Mo.  Dans ce
test,  le facteur déterminant reste le nombre de connexion concurrente néanmoins
nous verrons que le test n'est pas si propice à faire des conclusions.

Nous ferons  un test sur  la charge que  peut supporter le  serveur en utilisant
[-t] pour  limiter le traitement  à 1 seconde  et ainsi voir  combien de requête
nous pouvons traiter en ce temps.  Ce paramètre sera notre invariant par rapport
aux  autres.  Nous incrémenterons  ensuite le  nombre de  connexions simultanées
[-c] progressivement.  Enfin,  nous  aurons un nombre de  requête [-n] invariant
que nous comparerons au nombre de réponse reçu.

\subsection{Httperf}

Httperf est un logiciel \emph{single-threaded} créer  par HP un peu plus complet
et complexe que Apache Benchmark. Il est disponible à l'aide de:

\begin{lstlisting}
# aptitude install httperf
\end{lstlisting}

La  particularité de  Httperf est  qu'il augmente  progressivement le  nombre de
requête  pour  gérer   les  départs  à  froids  d'un   serveur  (comme  pour  un
échauffement progressif).  Un autre point le  caractérisant est une gestion bien
moins fine de la concurrence.  En effet, il peut arriver que toutes les requêtes
du premier \emph{shot} ne soient pas  terminées avant la deuxième seconde (si on
considère que vous  faites un \emph{shot} de [n] requête(s)  par seconde) ce qui
fait que  le prochain \emph{shot} de  requête contiendra peut  être les requêtes
non terminées du premier \emph{shot}.

Ici,  nous  augmenterons progressivement  l'option  [--rate]  pour  augmenter le
nombre de  connexion concurrente à lancer  (ce qu'on  nomme ici  le \emph{shot})
par seconde.  Ensuite,  nous  limiterons [--num-call] et  [--num-conn] qui sont
respectivement le nombre  de  requête  par  seconde  et  le nombre de connexions
concurrentes  (qui  correspondra  à la  borne  maximal  du  nombre  de connexion
simultané que le vont tester).

Ainsi,  dans l'utilisation de  Httperf,  pour que son test soit  presque le même
que celui de Apache Benchmark, [--rate] convergera vers [--num-call].

\subsection{Weighttp}

Weighttp  est un  outil \emph{multi-threaded}  créer par  l'équipe en  charge du
serveur HTTP Lighttpd.  Il n'est pas sur les dépôts officiels de Debian, il faut
donc faire une installation à la main:

\begin{lstlisting}
# aptitude install libev-dev
# git clone https://github.com/lighttpd/weighttp.git
# cd weighttp
# ./waf configure
# ./waf build
# ./waf install
\end{lstlisting}

Les options  de Weighttp sont  presque les mêmes  à ceci près  que l'option [-t]
dénote le nombre de processus légers pour l'exécution des requêtes.  Ensuite, le
logiciel nous informe du nombre de requête ayant eu une réponse en 1 seconde.

Il faut savoir  que Weighttp est plus  rapide que Apache Benchmark  (ce qui peut
être préjudiciable si on applique  la configuration pour limiter les TIME-WAIT).
Dans notre  contexte,  cela peut  être avec Apache  Benchmark le  meilleur outil
pour tester notre serveur HTTP.

\subsection{Outil d'abstraction}

Il  existe  enfin   des  outils  d'abstractions  testant   un  certain  contexte
d'utilisation   du  serveur   HTTP   cependant,   soit   ces   outils  utilisent
exclusivement l'un des outils  cité plus haut ce qui ne  permet pas de converger
les résultats,  soit leurs tests ne correspond pas à ce que nous cherchons, soit
il  ne s'intègre  pas à  notre contexte  physique  (à  savoir  un  serveur  et 2
clients).

On peut cependant parler d'\ref{autobench}, ou encore d'\ref{abc}.

\section{Le Test}

Les  critères  d'évaluation  d'un serveur  HTTP  sont  très  important  car nous
pourrions être amené  à évaluer autre chose que le  serveur HTTP.  Nous avons vu
que  nous  pouvions  tester  la vitesse  de  lecture  du  système  ou  tester le
scheduling de celui-ci si nous ne faisions pas attention.

Le critère  le plus probant  pour savoir la  vitesse pure  d'un serveur  HTTP et
dénotant rapidement  les limites  de celui-ci  est de  voir combien  de réponses
nous avons  eu du  serveur pour  N requêtes  pendant 1  seconde.  Les invariants
permettent de mettre en  valeur la rapidité du serveur et  de montrer combien de
requête le serveur est capable de traiter en une seconde.

Enfin,  un tel test  permet  de  mettre  en  valeur  des  points critiques où le
serveur ne sera plus  capable de répondre à toute les  requêtes et nous pourrons
ainsi  repérer par  dichotomie les  goulots d'étranglements  dans le  code (bien
entendu,  ceci  n'est  possible  que si  nous  avons  bien  abstrait  toutes les
limitations techniques et physiques liées au serveur).

Ce test est  aussi admis par la communauté  informatique comme représentatif des
performances entre les différents serveurs HTTP\footnotemark[1].

\footnotetext[1]{N2O                utilise                ce               test
(\url{https://github.com/5HT/n2o#why-erlang-in-web})  mais  GWAN  ce  base aussi
sur de tels critères.}

\end{document}
